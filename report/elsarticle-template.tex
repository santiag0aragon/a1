%%
%% Copyright 2007, 2008, 2009 Elsevier Ltd
%%
%% This file is part of the 'Elsarticle Bundle'.
%% ---------------------------------------------
%%
%% It may be distributed under the conditions of the LaTeX Project Public
%% License, either version 1.2 of this license or (at your option) any
%% later version.  The latest version of this license is in
%%    http://www.latex-project.org/lppl.txt
%% and version 1.2 or later is part of all distributions of LaTeX
%% version 1999/12/01 or later.
%%
%% The list of all files belonging to the 'Elsarticle Bundle' is
%% given in the file `manifest.txt'.
%%

%% Template article for Elsevier's document class `elsarticle'
%% with numbered style bibliographic references
%% SP 2008/03/01
%%
%%
%%
%% $Id: elsarticle-template-num.tex 4 2009-10-24 08:22:58Z rishi $
%%
%%
\documentclass[preprint,12pt,3p]{elsarticle}

%\documentclass[final,3p,times]{elsarticle}
%% Use the option review to obtain double line spacing
%% \documentclass[preprint,review,12pt]{elsarticle}

%% Use the options 1p,twocolumn; 3p; 3p,twocolumn; 5p; or 5p,twocolumn
%% for a journal layout:
%% \documentclass[final,1p,times]{elsarticle}
%% \documentclass[final,1p,times,twocolumn]{elsarticle}
%% \documentclass[final,3p,times]{elsarticle}
%% \documentclass[final,3p,times,twocolumn]{elsarticle}
%% \documentclass[final,5p,times]{elsarticle}
%% \documentclass[final,5p,times,twocolumn]{elsarticle}

%% if you use PostScript figures in your article
%% use the graphics package for simple commands
%% \usepackage{graphics}
%% or use the graphicx package for more complicated commands
%% \usepackage{graphicx}
%% or use the epsfig package if you prefer to use the old commands
%% \usepackage{epsfig}

%% The amssymb package provides various useful mathematical symbols
\usepackage{amssymb}
\usepackage{graphicx}
\usepackage[utf8]{inputenc}
\usepackage{color}
\usepackage[procnames]{listings}


\graphicspath{ {img/} }
%% The amsthm package provides extended theorem environments
%% \usepackage{amsthm}

%% The lineno packages adds line numbers. Start line numbering with
%% \begin{linenumbers}, end it with \end{linenumbers}. Or switch it on
%% for the whole article with \linenumbers after \end{frontmatter}.
%% \usepackage{lineno}

%% natbib.sty is loaded by default. However, natbib options can be
%% provided with \biboptions{...} command. Following options are
%% valid:

%%   round  -  round parentheses are used (default)
%%   square -  square brackets are used   [option]
%%   curly  -  curly braces are used      {option}
%%   angle  -  angle brackets are used    <option>
%%   semicolon  -  multiple citations separated by semi-colon
%%   colon  - same as semicolon, an earlier confusion
%%   comma  -  separated by comma
%%   numbers-  selects numerical citations
%%   super  -  numerical citations as superscripts
%%   sort   -  sorts multiple citations according to order in ref. list
%%   sort&compress   -  like sort, but also compresses numerical citations
%%   compress - compresses without sorting
%%
%% \biboptions{comma,round}

% \biboptions{}


\begin{document}

\begin{frontmatter}

\title{Assignment 1: Mixnets}

\author{Santiago Aragón}
\address{s.e.aragonramirez@student.utwente.nl}

\author{Owais Ahmed}
\address{o.ahmed@student.utwente.nl}
\address{University of Twente}

%\begin{abstract}
%Text of abstract. Text of abstract. %Text of abstract. Text of abstract. %Text of abstract.
%\end{abstract}


\end{frontmatter}

%%
%% Start line numbering here if you want
%%
% \linenumbers

%% main text



\section*{Assignment 1}
\begin{flushleft}
\textbf{Part A:}
We developed an application mixnets.py [Appendix A] to send messages via the three mix nodes and the cache node that forwards the individual messages to the recipient.
\newline

\textbf{Part B:}
We sent a message to TIM from the send message method as shown below in mixnets.py application [Appendix A].
\begin{verbatim}send_message('TIM     ','s1750542  and  s1736574') \end{verbatim}

\textbf{Part C:}

\begin{figure}[h]
\caption{Frequency of messages received against time in the same second.}
\centering
\includegraphics[width=\textwidth]{one_c}
\end{figure}

We parsed the cache log to analyse the individual messages and performed frequency analysis as shown in Figure: 1 by counting the number of messages received in a particular second and plotted the results in a bar chart graph. We observed that mostly seven messages were received in the cache log in a particular second, however in certain instances, the average of consecutive messages received in two seconds was seven. We therefore came to the conclusion that $n_C$ is 7, and since we know that the threshold of $n_A$ = $n_B$ = $n_C$, it implies that $n_A$ , $n_B$ and $n_C$ is 7.

\section*{Assignment 2}
\textbf{Part A:}
\newline

We sent individual messages one by one with a short time delay and observed the output via the cache log by parsing the message fields. We observed that the messages forwarded by MIX C to the CACHE NODE had a frequency of mostly 6 or 9. We learned that the sum of messages received in consecutive seconds was always a factor of 3, as shown in Figure: 2. We therefore, we came to the conclusion that the threshold of $n_C$ is 3.
\newline
\begin{figure}[h]
\caption{Frequency of messages received against time in the same second.}
\centering
\includegraphics[width=\textwidth]{second}
\end{figure}
\newline
We kept a count of the messages entering the second mixnet via MIX A and the messages received in the cache log after reaching the threshold $n_C$. After the first 8 messages passed through MIX A, only 6 messages were displayed in the cache log. Since we know that the threshold of MIX B has to be at least more than $2n_C$ and less than $3n_C$. This implies that the threshold of MIX B is 6 $<$ $n_B$ $<$ 9. Furthermore, we have only inject 8 messages and we know that $n_A\geq 1$ , therefore we conclude that the threshold $n_B$ is 7.
\newline
We kept a count of the messages entering the second mixnet via MIX A and the messages received in the cache log after reaching the threshold $n_C$. After the first 8 messages passed through MIX A, only 6 messages were displayed in the cache log, that denotes that 1 $\leq$ $n_A, n_B$ $\leq$ 8. We sent more messages one by one until a second batch of 6 messages were received in the cache log. We noted that after sending 6 more messages, another batch of 6 messages was received in the cache log, this further helped us to analyse that MIX A always accepted even number of messages and the least common factor of the input messages to obtain an output was always 2. We therefore assumed that the threshold of  $n_A$ is 2.
\newline
Considering $n_A$ is 2 and $n_C$ = 3, we kept a count of the messages sent into the mixnet that were not received by the cache log and noted the pending messages still in the queue for not reaching the threshold. We observed that $n_B$ required 6 more messages to reach a threshold of 7 keeping  track of unreceived messages, therefore the threshold $n_B$ is 7.
\newline


Considering $n_A$ is 2 and $n_C$ = 3, we kept a count of the messages sent into the mixnet that were not received by the cache log and noted the pending messages still in the queue for not reaching the threshold. We observed that $n_B$ required 6 more messages to reach a threshold of 7 keeping  track of unreceived messages, therefore the threshold $n_B$ is 7.

Considering $n_A$ is 2 and $n_C$ = 3, we kept a count of the messages sent into the mixnet that were not received by the cache log and noted the pending messages still in the queue for not reaching the threshold. We observed that $n_B$ required 6 more messages to reach a threshold of 7 keeping  track of unreceived messages, therefore the threshold $n_B$ is 7.

Considering $n_A$ is 2 and $n_C$ = 3, we kept a count of the messages sent into the mixnet that were not received by the cache log and noted the pending messages still in the queue for not reaching the threshold. We observed that $n_B$ required 6 more messages to reach a threshold of 7 keeping  track of unreceived messages, therefore the threshold $n_B$ is 7.

\begin{center}
Threshold of $n_A$ is 2

Threshold of $n_B$ is 7

Threshold of $n_C$ is 3
\end{center}




\textbf{Part B:}

\section*{Assignment 3}
\textbf{Part A:}

To deanonymize the party that is communicating with TIM we launch a n-1 attack. We recall that we are a global active attacker with insert capabilities, namely, we have access to two logs one at the entrance (client log) of the mixnet and one at very end (cache log).

We have find the threshold of every MIX in section~\ref{sec2} and we know that the first 2 batches are triggered with 8 messages. Thus, after starting the mixnet we insert 7 messages and wait for a message coming from some mixnet user. The very first message that arrive will push 6 messages to the cache, to which we have access through the logs. By comparing the client log and the cache log we are able to deanonymize the first message that enters the mixnet after our first 7 messages. We define a function called \textit{n\_1\_attack} where we start the mixnet, perform a n-1 attack and repeats until the deanonymized user is the one that communicates with Tim.




\textbf{Part B:}
....
\newline

\end{flushleft}

%% appendix sections are then done as normal sections
\appendix



\begin{lstlisting}[language=Python]
def n_1_a():
    start(3)
    sleep(.05)
    send_message('ME ', '-'*7)
    send_message('ME ', '-'*7)
    send_message('ME ', '-'*7)
    send_message('ME ', '-'*7)
    send_message('ME ', '-'*7)
    send_message('ME ', '-'*7)
    send_message('ME ', '-'*7)
    log = parseClientLog()
    if log == '':
        message_sent = 7
        while cache_num() < 6:
            pass
        stop()
        sleep(2)
        if not check_for_tim() and not_me() is not  None:
            rec = not_me()
            sen = first_client()
            print '%s is communicating with %s' %(sen, rec)
            n_1_a()
        elif not_me() is  None:
            n_1_a()
        else:
            rec = not_me()
            sen = first_client()
            print '%s is communicating with %s' %(sen,rec)
    else:
        n_1_a()
\end{lstlisting}
% \section{Mixnet 1}
% \section{Mixnet 2}
% \section{Mixnet 3}
\label{appendix-sec1}
%% References
%%
%% Following citation commands can be used in the body text:
%% Usage of \cite is as follows:
%%   \cite{key}         ==>>  [#]
%%   \cite[chap. 2]{key} ==>> [#, chap. 2]
%%
%% References with bibTeX database:

\bibliographystyle{elsarticle-num}
% \bibliographystyle{elsarticle-harv}
% \bibliographystyle{elsarticle-num-names}
% \bibliographystyle{model1a-num-names}
% \bibliographystyle{model1b-num-names}
% \bibliographystyle{model1c-num-names}
% \bibliographystyle{model1-num-names}
% \bibliographystyle{model2-names}
% \bibliographystyle{model3a-num-names}
% \bibliographystyle{model3-num-names}
% \bibliographystyle{model4-names}
% \bibliographystyle{model5-names}
% \bibliographystyle{model6-num-names}
\bibliography{sample}


\end{document}

%%
%% End of file `elsarticle-template-num.tex'.
